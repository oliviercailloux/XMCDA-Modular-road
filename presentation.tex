\RequirePackage[l2tabu, orthodox]{nag}
\RequirePackage{silence}
\documentclass[french,english]{beamer}
\input{preamble/packages}
\input{preamble/math_basics}
\input{preamble/math_mine}
\input{preamble/redac}
%\input{preamble/draw}
\input{preamble/acronyms}

\setbeamertemplate{headline}[singleline]
%\setbeamertemplate{footline}[authortitle]

\title[Modular XMCDA]{A proposal for modular XMCDA}
\subject{XMCDA}
\keywords{key, koy}
\author{Olivier Cailloux}
\institute[LAMSADE]{LAMSADE, Université Paris-Dauphine}
\date{\formatdate{20}{9}{2017}}

\begin{document}
\begin{frame}[plain]
	\tikz[remember picture,overlay]{
		\path (current page.south west) node[anchor=south west, inner sep=0] {
			\includegraphics[height=1cm]{LAMSADE95.jpg}
		};
		\path (current page.south) ++ (0, 1mm) node[anchor=south, inner sep=0] {
			\includegraphics[height=9mm]{Dauphine.jpg}
		};
		\path (current page.south east) node[anchor=south east, inner sep=0] {
			\includegraphics[height=1cm]{PSL.png}
		};
		\path (current page.south) ++ (0, 4em) node[anchor=south, inner sep=0] {
			\scriptsize\url{https://github.com/oliviercailloux/XMCDA-Modular-road}
		};
	}
	\titlepage
\end{frame}
\addtocounter{framenumber}{-1}

\section{My goal}
\subsection{Focus}
\begin{frame}
	\frametitle{My goal (broadly)}
	\begin{itemize}
		\item Improve XMCDA
		\item Long term goal!
		\item Compatible with evolutions to come (XMCDA v3)
		\item This is not a criticism of existing work
		\item \emph{I}’ll (try to) take care of it
		\item Comments / Help welcome
	\end{itemize}
\end{frame}

\begin{frame}
	\frametitle{Outline}
	\tableofcontents[hideallsubsections, sectionstyle=shaded/show]
\end{frame}

\AtBeginSection{
	\begin{frame}
		\frametitle{Outline}
		\tableofcontents[currentsection, hideallsubsections]
	\end{frame}
}

\section{Overview}
\begin{frame}
	\frametitle{My goal (more specifically)}
	\begin{itemize}
		\item We want to let \acp{WS} communicate
		\item Communication requires standard language
		\item Common language ≠ constraints specific to WS
		\item Currently: non standard encoding of specific constraints (diviz-like)
		\item Not taken into account in your programming language
		\item SOAP calls allowed, but does not embrace standards
		\item My goal 1: improve compatibility with standards
		\item My goal 2: ease development
		\item My constraint: compatibility with XMCDA
	\end{itemize}
\end{frame}

\section{Current state}
\begin{frame}
	\frametitle{Example of \acp{WS}}
	\ac{WS} 1
	\begin{itemize}
		\item I want a non-empty list of alternatives each having a name and a value
		\item Example input (conceptually) : $\set{("a1", 3), ("a2", 10), ("blah", 4)}$
		\item I provide as an output an ordered list of alternatives
		\item Example output: $("a2", "blah", "a1")$
	\end{itemize}
	\ac{WS} 2
	\begin{itemize}
		\item Input: an ordered list of alternatives; evaluations for each of these alternatives; a value function
		\item Output: binary (is the list ranked according to the value function?)
	\end{itemize}
\end{frame}

\begin{frame}
	\frametitle{Desired}
	\begin{itemize}
		\item In example: Common encoding of “ordered list of alternatives”
		\item In general: Common encoding of other inputs and outputs: “list of alternatives each having a name and a value”, …
		\item With low redundancy: only one way of specifying each possible input type
		\item Common language, but each WS has specific inputs and outputs
	\end{itemize}
%		\item Developer is helped with automatic check of type constraints satisfaction
%		\item Declaration of constraints in a standard language: 
\end{frame}

\begin{frame}
	\frametitle{Current approach}
	Common language
	\begin{itemize}
		\item Each \ac{WS} must accept and produce XMCDA
		\item Specified with a schema
		\item Version 3 in preparation
		\item Improvements done about “low redundancy”
	\end{itemize}
	Specificities of \acp{WS}
	\begin{itemize}
		\item Not every \ac{WS} accepts every XMCDA valid input
		\item Each \ac{WS} specifies its supplementary constraints in a diviz web descriptor
		\item Diviz uses this for its user interface
		\item Developers manually check their input
	\end{itemize}
\end{frame}

\begin{frame}
	\frametitle{Some details}
	Common language
	\begin{itemize}
		\item XMCDA schema specifies the grammar of valid XMCDA encoding
		\item Very permissive (out of necessity): has to accept inputs and outputs that are valid for at least one \ac{WS}
		\item Valid iff contains alternatives OR criteria OR value functions OR …
	\end{itemize}
	\newlength{\XMSchLeft}
	\setlength{\XMSchLeft}{6cm}
	\begin{minipage}[c]{\XMSchLeft}
		Specific constraints
		\begin{itemize}
			\item Developer specifies input type
			\item Through Diviz descriptor (\href{http://www.decision-deck.org/ws/_downloads/description-wsDD22.xml}{example})
		\end{itemize}
	\end{minipage}%
	\begin{minipage}[c]{\the\columnwidth - \XMSchLeft}
		\includegraphics[width=\columnwidth]{diviz-J-MCDA-components.png}
	\end{minipage}
\end{frame}

\begin{frame}
	\frametitle{Some details: simplify developers work}
	Also: simplify developers work
	\begin{itemize}
		\item Diviz framework cuts this input into pieces: alternatives; value functions; …
		\item Developer programs a \ac{WS} by simply reading from different inputs (one input per type)
	\end{itemize}
	But…
	\begin{itemize}
		\item Tendency to mandate use of library to guarantee constraints satisfaction beyond schema valid files
	\end{itemize}
\end{frame}

\section{Modular XMCDA}
\begin{frame}
	\frametitle{Goals: More standard conforming}
	\begin{itemize}
		\item Use XML schemas to constrain every \acp{WS}
		\item SOAP calls with schema constrained input
		\item Automatic discovery of compatible \acp{WS} possible
		\item Automatic invocation and GUI generation possible
		\item Better integration with enterprise infrastructure
		\item Decentralisation: provide XMCDA \ac{WS} with industry standard tools
		\item Easier interface with existing services (example: solve LP problem)
	\end{itemize}
\end{frame}

\begin{frame}
	\frametitle{Goals: Ease developer work}
	\begin{itemize}
		\item Give guarantee: input type has been checked against some specific grammar
		\item Example: I know I receive a list of …
		\item Possible to use libraries to transform XML to language specific types
	\end{itemize}
\end{frame}

\begin{frame}
	\frametitle{Constraints}
	\begin{itemize}
		\item Interface various \acp{WS}
		\item Reuse global encoding standard (XMCDA v2 / v3)
		\item Compatibility with existing \acp{WS}
		\item As easy to use for non-XML aware developers
	\end{itemize}
\end{frame}

\begin{frame}
	\frametitle{Approach}
	\begin{itemize}
		\item XMCDA-Modular (XM) Standard provides a set of types
		\item Called XM types
		\item Example: set of alternatives, value function, …
		\item Each \ac{WS} has a specific schema
		\item This schema is composed of a list of entries
		\item Some of them are XM types
	\end{itemize}
	Compare to: one schema for every \ac{WS}
\end{frame}

\begin{frame}
	\frametitle{Reuse}
	\begin{itemize}
		\item The single types could be borrowed from XMCDA v. x
		\item Some types could even be made more constrained
		\item Developer may provide its own type when too specific for deserving being in the standard
		\item Developers may re-use types from the industry, e.g. LP
		\item Compatibility with existing \acp{WS}: have to develop translators (per type)
	\end{itemize}
\end{frame}

\begin{frame}
	\frametitle{Developer POV}
	\begin{itemize}
		\item Instead of Diviz descriptor, has to provide a schema
		\item More difficult!
		\item Possible to help with tools (to be developed!)
		\item Possible conversion with Diviz descriptor files
		\item Leverage existing tools for easy / automatic parsing of input / output
		\item[⇒] To be developed as well!
		\item Provides standard, more general functionality equivalent to existing manually programmed libraries, with support for languages to come
	\end{itemize}
\end{frame}

\begin{frame}
	\frametitle{Infrastructure}
	\begin{itemize}
		\item Leverage WSDL, language for SOAP WS description
		\item Provides standard, more general equivalent to Diviz descriptor
		\item Extend Diviz to account for standard (WSDL) descriptors suppl to Diviz-like
	\end{itemize}
\end{frame}

\section{Implementation}
\begin{frame}
	\frametitle{Implementation}
	\begin{itemize}
		\item Course at Dauphine, MIAGE M2
		\item Some of these students know the prerequisites
		\item Could be interested in (parts of) this as student projects
	\end{itemize}
\end{frame}

\begin{frame}
	\frametitle{Proposed course of action}
	\begin{itemize}
		\item You comment
		\item We discuss
		\item I (try to) get parts of the idea implemented
		\item Using XMCDA-Modular \href{https://github.com/xmcda-modular/schema}{schema} types to get started
		\item To be developed progressively!
		\item I keep you posted
		\item Possibly: request some resources for online hosting (later!)
		\item This proposal is compatible with any improvements to the XMCDA schema
		\item Don’t hold your breath!
	\end{itemize}
\end{frame}

\begin{frame}[plain]
	\addtocounter{framenumber}{-1}
	\begin{center}
		\huge
		\textit{Thank you for your attention!}
	\end{center}
\end{frame}

\appendix
\AtBeginSection{
}

\end{document}

\clearpage\pdfbookmark[2]{\refname}{\refname}
\begin{frame}[allowframebreaks]
	\frametitle{\refname}
 	\bibliography{zotero}
\end{frame}

\clearpage\pdfbookmark{License}{License}
\begin{frame}[plain]
	\frametitle{License}
	This presentation, and the associated \LaTeX{} code, are published under the \href{http://opensource.org/licenses/MIT}{MIT license}. Feel free to reuse (parts of) the presentation, under condition that you cite the author.
	
	Credits are to be given to \href{http://www.lamsade.dauphine.fr/~ocailloux/}{Olivier Cailloux}, Université Paris-Dauphine.
\end{frame}
\addtocounter{framenumber}{-1}
\end{document}

\begin{frame}
	\frametitle{\subsecname}
	\begin{itemize}
		\item 
	\end{itemize}
\end{frame}

